\documentclass[a4paper]{article}

\usepackage[applemac]{inputenc}
\usepackage{ngerman}
%\usepackage{tipa}
\usepackage{graphics,latexsym}
\usepackage{longtable}
\usepackage{rotating}

\setlength\parskip{\smallskipamount} \setlength\parindent{0pt}
\newcommand{\noun}[1]{\textsc{#1}}
%\renewcommand{\section}{\normalfont}
%\large\sffamily\bfseries}

\author{Fabian Steeg}
\date{\today}
\title{\\ \\[10ex]\Large{Universit"at zu K"oln \\ Institut f"ur Linguistik
}\\[5ex] \large{Proseminar ``Theorien und Modelle''\\ bei Prof. Hans-J"urgen
Sasse}\\[10ex]
\LARGE{\noun{Simon C. Diks ``Functional Grammar''}}\\[5ex]}

%\\[5ex]

% doppelzeilig -- fuer 1 1/2 den Wert 1.3 nehmen!
\linespread{1.3}

\begin{document}
\begin{titlepage}
\maketitle
\thispagestyle{empty}

\newpage

\tableofcontents \listoffigures
%\listoftables
\thispagestyle{empty}
\end{titlepage}



\section{Functional Grammar}

``Functional Grammar'' (FG) wurde Ende der 1970er Jahre von Simon C. Dik in
Amsterdam entwickelt -- ausdr"ucklich als Gegenmodell zum Standard-Modell der
Tranformationsgrammatik von Noam Chomsky. Es ist das einzige vollst"andige
solche Gegenmodell, das ausserhalb des MIT (wo Chomsky das Standardmodell
ent\-wickelt hatte) ent\-standen ist. Nach dem Tod Diks 1995 wurde die Theorie
vor allem durch seinen Mitarbeiter Kees Hengeveld weiterentwickelt und ist in
ihrer heutigen Form der urspr"unglichen Formulierung noch sehr nah.

\section{Allgemeine Annahmen "uber Sprache} \label{functional} \label{formal}

Die zentrale Annahme Diks "uber Sprache ist ihr zweckgebundener Charakter als
Mittel zur Kommunikation. Dik r"uckt damit die Funktion der Sprache in den
Mittelpunkt. In diesem Sinne ist die Bezeichnung ``Functional Grammar'' zu sehen:
Ein sprachliches Modell, das von der Funktion der Sprache, statt von ihrer
"ausseren Form ausgeht. Mit dieser zentralen Annahme fordert Dik eine Abkehr
von der fr"uher h"aufig angewandten heuristischen Reduktion der Ausblendung der
Pragmatik. Konkret ist jedoch in der Behandlung der Pragmatik bei Dik nicht die
allgemeine Pragmatik im Sinne von Sprechakten und Sprache als Handlung gemeint,
sondern der Bereich der Diskurspragmatik, im Wesentlichen also das Verh"altnis
der Informationsstruktur eines sprachlichen Ausdrucks zu seiner Realisierung,
etwa bei der Behandlung von Topik und Fokus (Dik 1991:267ff.).

Der von Dik beschriebene Grammatikformalismus ist in diesem Sinne
pragmatikbasiert. Die n"achst wichtigste sprachliche Ebene ist aus Diks Sicht
die Semantik, die -- selbst von der Pragmatik beeinflusst -- ihrerseits
Einfluss auf die Syntax hat. Ein Beispiel f"ur eine solche Beeinflussung der
Syntax w"are etwa eine Aktiv-Passiv-Alternation, die von den semantischen
Rollen der Mitspieler in der "Ausserung bestimmt wird, und in diesem Sinne
semantisch motiviert ist.

Zugleich ist FG ein formales Modell, da es Methoden der
formalen Semantik -- etwa Pr"adikatenlogik -- verwendet und den Anspruch der
Implementierbarkeit als Computerprogramm und damit der Testbarkeit erhebt.

\section{Vorstellungen "uber Grammatik}

\subsection{Grundlegender Aufbau des Grammatikformalismus}

Der Grammatik-Formalismus der FG besteht im wesentlichen aus
der Beschreibung abstrakter Ausdr"ucke, der \emph{Underlying Clause Structures}
(UCS), die schrittweise aus Pr"adikaten und Termen gebildet werden und die
durch Ausdrucksregeln zu konkreten sprachlichen "Ausserungen in Bezug gesetzt
werden oder diese erzeugen.

\subsection{Aufbau der Underlying Clause Structure}

\subsubsection{Pr"adikate}

Die UCS werden aus Pr"adikaten und Termen gebildet. Einige elementare
Pr"adikate und Terme sind Teil des Lexikons, andere werden aus diesen
elementaren Pr"adikaten und Termen erstellt (durch \emph{predicate formation}
und \emph{term formation}, siehe Abbildung \ref{fg} auf Seite \pageref{fg}). So
w"are das Pr"adikat f"ur \emph{throw back} ein aus den elementaren Pr"adikaten
f"ur \emph{throw} und \emph{back} abgeleitetes Pr"adikat. Alle Pr"adikate und
Terme zusammen bilden den Fundus (\emph{fund}) einer Sprache.

Pr"adikate sind Ausdr"ucke f"ur Eigenschaften oder Relationen. Es handelt sich
hier um Pr"adikate im Sinne der Pr"adikatenlogik, nicht um die grammatische
Relation des Pr"adikates aus der lateinischen Schulgrammatik. In diesem Sinne
sind nicht nur Verben Pr"adikate, sondern alle Inhaltsw"orter einer Sprache. So
ist ``haus(x)'' etwa ebenso ein Pr"adikat wie ``schlagen(x,y)''.

Ein Unterschied der Pr"adikate in der FG zur klassischen Pr"adikatenlogik ist
die Verwendung sogenannter Restriktoren. Wenn in der FG Pr"adikate
zusammengesetzt werden, geschieht dies durch die Verwendung dieser
Restriktoren, geschrieben als ``:'', etwa in der Form
``japanisch(x):buddhistisch(x)''. Dies l"asst sich paraphrasieren mit ``Die
Menge der x, f"ur die gilt: x ist japanisch, eingeschr"ankt auf die Menge der
x, f"ur die gilt: x ist buddhistisch''. Die entsprechende pr"adikatenlogische
Form w"are ``japanisch(x) \& buddhistisch(x)'', wobei das ``\&'' ein
pr"adikatenlogisches ``UND'' ist. Der entsprechende Sachverhalt ist in beiden
F"allen gleich. Der Unterschied besteht darin, dass das pr"adikatenlogische
``\&'' umkehrbar ist, dass also ``japanisch(x) \& buddhistisch(x)'' aquivalent
ist zu ``buddhistisch(x) \& japanisch(x)''. Bei den Restriktoren ist dies nicht
der Fall und sie sind damit in der Lage, den Unterschied der sprachlichen
"Ausserungen \emph{Der japanische Buddhist} und \emph{Der buddhistische
Japaner} zu erfassen (Dik 1997, Kap. 6.2).

Pr"adikate sind stets Teil eines Pr"adikatrahmens, der die Eigenschaften des
Pr"adikats beschreibt. Ein Beispiel f"ur den Pr"adikatrahmen eines transitiven
Verbs w"are etwa:

\begin{quote}
throw[V](x1:\textless animate \textgreater (x1))$_{Agent}$ (x2:\textless
concrete\textgreater (x2))$_{Goal}$ (x3)$_{Direction}$
\end{quote}

Zun"achst erscheint die Wortform (throw), anschlie"send die Wortart (V). Im
Folgenden sind die Argumentpositionen des Verbs beschrieben. Das Argument in
der Mitspielerposition mit der semantischen Rolle des Agens muss belebt
(animate) sein, der vom Sachverhalt betroffene Mitspieler
(Goal\footnote{Allgemein hat sich f"ur die Rolle, die Dik \emph{Goal} nennt die
Bezeichnung \emph{Patient} bzw. \emph{Patiens} durchgesetzt.}) -- hier der
geworfene Gegenstand -- muss konkret (concrete) sein und das dritte Argument
(mit der semantische Rolle Location) unterliegt keiner solchen
Selektionsbeschr"ankung.

Zu solchen nuklearen Pr"adikaten k"onnen nun die fakultativen sogenannten
Satelliten hinzukommen, die Positionen einnehmen, die nicht vom Pr"adikatrahmen
spezifiziert werden, etwa zu einer zeitlichen Pr"azisierung des Pr"adikats mit
Hilfe von W"ortern wie \emph{gestern} oder \emph{bald}. Einen solchen um
Satelliten erweiterten Pr"adikatrahmen nennt Dik einen erweiterten
Pr"adikatrahmen (\emph{extended predicate frame}). Dies ist im Kasten
``Predicate-Frames'' in Abbildung \ref{fg} auf Seite \pageref{fg} schematisch
dargestellt.

\subsubsection{Terme}

Der zweite wesentliche Bestandteil einer UCS sind neben Pr"adikaten die Terme.
Formal sind Terme die Argumente der Pr"adikate, semantisch sind es Ausdr"ucke,
die Entit"aten referenzieren\footnote{Genau genommen schreibt Dik (1991:255),
dass Terme den Adressaten instruieren, eine Entit"at zu identifizieren, die dem
Profil des Terms entspricht.}. Beispiele f"ur Terme w"aren \emph{Das Haus} oder
\emph{Die lila Plastikt"ute}. Es existieren nur sehr wenige elementare Terme,
so sind lediglich Eigennamen und Personalpronomina als elementare Terme
vorhanden, andere Terme, wie \emph{Die lila Plastikt"ute} werden aus
Pr"adikaten erstellt. Terme sind also die Entit"aten, die durch ein Pr"adikat
zueinander in Beziehung gesetzt werden.

Eine Pr"adikation, die zwei Terme (\emph{the garden} und \emph{the dog})
enth"alt w"are z.B.:

\begin{quote}
present: (definite singular x1: garden [N])$_{Location}$ (definite singular x2:
dog [N])
\end{quote}

\subsubsection{Ebenen in der UCS}

Innerhalb der UCS werden in Form von Funktionen drei verschiedenen Ebenen
unterschieden:

\begin{enumerate}
  \item {"Ausserungssituation: Ebene der pragmatischen Funktionen wie Topik und
  Fokus.}
  \item {Mitspielerebene: Ebene der semantischen Funktionen wie Agent und Goal
  (Patiens).}
  \item {Ebene der Perspektive: Ebene der syntaktischen Funktionen Subjekt und
  Objekt.}
\end{enumerate}

In diesem Sinne nehmen einzelne Elemente einer sprachlichen "Ausserung auf den
verschiedenen Ebenen verschiedene Kategorien an. In dem Satz \emph{Peter kauft
ein Eis} etwa ist \emph{Peter} zugleich Agens, Topik und Subjekt, w"ahrend
\emph{Eis} zugleich Goal (Patiens), Fokus und Objekt ist.

Der sprachliche Ausdruck, der durch die Ausdrucksregeln auf die UCS

\begin{quote}
present: (definite singular x1: garden [N])$_{Location}$ (definite singular x2:
dog [N])
\end{quote}

bezogen (oder in einer Implementierung des Formalismus auch aus der UCS
erzeugt) werden kann ist so noch nicht eindeutig. Der UCS entspricht etwa die
"Ausserung \emph{The dog is in the garden}. In einer bestimmten
"Ausserungssituation -- etwa in einer Aufz"ahlung der f"ur einen Einbruch zu
"uberwindenden Hindernisse -- w"are aber folgende "Ausserung denkbar, die
ebenfalls mit der UCS "ubereinstimmt: \emph{There is the dog in the garden}.
Dieses Beispiel verdeutlicht die M"oglichkeiten, die eine Ber"ucksichtigung der
pragmatischen Ebene bietet, denn durch die Kennzeichnung des
Aufz"ahlungscharakters ist es m"oglich, die beiden sprachlichen "Ausserungen in
der zugrunde liegenden Struktur zu unterscheiden (Dik 1997, Kap. 8.7.2).

Die Unterscheidung der Ebene der semantischen Rollen, d.h. der Mitspieler und
der syntaktischen (grammatischen) Relationen erm"oglicht die Beschreibung
syntaktischer Alternationen wie der Passivierung ohne dass dabei die eine Form
aus der anderen abgeleitet werden m"usste. So gibt es in einem Aktivsatz eine
"Ubereinstimmung von Subjekt und Agens, w"ahrend bei einer "Ubereinstimmung von
Subjekt und Goal (Patiens) in der UCS dieser ein Passivsatz entspr"ache.

\subsubsection{Operatoren auf Pr"adikationen}

Wenn wie beschrieben die Terme in die Pr"adikatrahmen eingesetz wurden, haben
wir eine Pr"adikation, die die vollst"andige Proposition oder den Sachverhalt
(\emph{State of Affair}, SoA) des Satzes enh"alt, jedoch noch nicht weiter
spezifiziert ist. Dazu werden nun Operatoren auf die gesamte Pr"adikation
angewandt, etwa in der UCS oben der Operator ``present'', der selbst wieder als
ein Pr"adikat mit der vollen Pr"adikation als Argument gesehen werden kann.
Ebenso werden in diesem Schritt Operatoren zum Modus (etwa Interrogativ oder
Deklarativ) eingef"ugt.

Diese nun voll spezifizierte Pr"adikation wird schlie"slich mit Hilfe von
Ausdrucksregeln zu Form, Reihenfolge und Intonation spezifiziert und damit zu
einer konkreten sprachlichen "Ausserung in Bezug gesetzt (bei der Beschreibung)
bzw. in eine solche umgewandelt (bei der Generierung).

\subsection{Zusammenfassung des Grammatikformalismus} \label{monostratal} \label{deszendent}

Es handelt sich damit bei FG um ein monostratales Modell,
da zwar zwischen der UCS und den sprachlichen Ausdr"ucken unterschieden wird
und diese durch Ausdrucksregeln aufeinander bezogen werden, jedoch werden keine
verschiedenen Ebenen angenommen, auf denen konkrete sprachliche "Ausserungen
stehen, so sind etwa keine syntaktischen Derivationsmechanismen vorhanden. In
diesem Sinne findet die Bildung der sprachlichen "Ausserungen schrittweise
innerhalb einer Prozesskette, auf einer einzigen Ebene statt, wie in Abbildung
\ref{fg} auf Seite \pageref{fg} deutlich wird, die den Aufbau einer FG zeigt.

Die Pragmatikorientiertheit macht FG zu einem deszendenten Grammatikmodell, das
von der Gesamtsituation ausgeht, in der eine "Ausserung get"atigt wird, im
Gegensatz zu einem aszendenten Grammatikmodell, das von den kleinsten Teilen
ausgeht, etwa von der Phonologie "uber die Morphologie zur Syntax.

\begin{figure}
\begin{center}
\includegraphics{fg.png}
\end{center}
\caption{Aufbau einer FG (aus Dik 1991:250)} \label{fg}
\end{figure}

\section{Behandlung der Daten} \label{induktiv}

Die Bedeutung von sprachlichen Daten setzt Dik im Allgemeinen sehr hoch an:

\begin{quote}
``Whenever there is some overt difference between two constructions X and Y,
start out on the assumption that this difference has some kind of functionality
in the linguistic system'' (Dik 1997, Kap. 1.6).
\end{quote}

Damit hat FG einen induktiven Charakter, da sie "ahnlich dem
Bloomfieldschen Deskritptivismus von konkreten sprachlichen Daten ausgeht, im
Gegensatz zu einem deduktiven Modell wie der Generativen Grammatik nach
Chomsky, wo eine ideale, vom konkreten Sprachgebrauch abstrahierte
Sprachkompentenz im Mittelpunkt der Theorie steht.

In den zentralen Bereichen Pragmatik und Semantik ist die FG
vor allem auf die Befragung von Informanten (Elizitierung) sowie die
Konsultation der eigenen muttersprachlichen Einsichten (Introspektion)
angewiesen.  Andere Quellen wie Experimente oder Korpora sind nicht ohne
weiteres\footnote{Eine Generierung von semantischem Wissen w"are eventuell
durch eine automatische Verarbeitung von Korpora, etwa zur Ermittlung
paradigmatischer oder syntagmatischer Relationen m"oglich.} zur Ermittlung
semantischen Wissens (etwa f"ur die Selektionsbeschr"ankungen in
Pr"adikatrahmen) verwendbar.

Zur Evaluierung des Gesamtmodells k"onnen dagegen auch in der FG Korpora und
damit spontansprachliche Daten verwendet werden, etwa zur "Uberpr"ufung, ob
"Ausserungen in Korpora durch den FG-Formalismus beschrieben werden k"onnen.

\section{Anspruch des Modells} \label{kriterien} \label{mentalistisch} \label{universalistisch}

Die Zielsetzung der FG ist sehr umfassend, Dik (1997, Kap. 1) formuliert
folgende zentrale Frage: ``How does the natural language user (NLU) work?''.
Diese Fragestellung kennzeichnet FG klar als Modell mit mentalistischem
Anspruch.

Dik identifiziert im Anschluss an die Formulierung dieser zentralen
Fragestellung f"unf F"ahigkeiten des NLU, die essentielle Rollen f"ur die
menschliche Kommunikation spielen:

\begin{enumerate}
  \item {\emph{linguistic capacity}: F"ahigkeit zur Produktion und
  Interpretation sprachlicher Ausdr"ucke.}
  \item {\emph{epistemic capacity}: F"ahigkeit zu Aufbau und Verwaltung einer
  Wissensbasis, die zur Sprachverarbeitung genutzt wird.}
  \item {\emph{logical capacity}: Die F"ahigkeit, Schlussfolgerungen aus dem
  verf"ugbaren Wissen zu ziehen.}
  \item {\emph{perceptual capacity}: F"ahigkeit, seine Umwelt wahrzunehmen und
  bei der Sprachverarbeitung zu ber"ucksichtigen.}
  \item {\emph{social capacity}: F"ahigkeit, die Situation bei der
  Sprachverarbeitung mit zu ber"ucksichtigen.}
\end{enumerate}

Dar"uber hinaus formuliert Dik in Anspielung auf die von Chomsky geforderten
drei Ad"aquatheitskriterien der Beschreibungs-, Erkl"arungs- und
Beobachtungsad"aquatheit drei eigene, v"ollig andere Ad"aquatheitskriterien:

\begin{enumerate}
  \item {Pragmatische Ad"aquatheit: Direkte Folge der Grundannahme, dass
  Sprache ein Mittel zur Kommunikation darstellt.}
  \item {Psychologische Ad"aquatheit: Erkenntnisse aus der
  psycho\-linguistischen Forschung zu Sprach\-erwerb, -ver\-ar\-bei\-tung und
  -inter\-preta\-tion m"ussen ber"ucksichtigt werden.}
  \item {Typologische Ad"aquatheit: Die Theorie soll auf Sprachen von
  unterschiedlichem typologischem Status anwendbar sein.}
\end{enumerate}

Insbesondere durch den Anspruch der typologischen Offenheit erh"alt das Modell
einen stark beschreibungsorientierten Charakter, da es eine solche Offenheit zu
einem universellen Beschreibungswerkzeug machen w"urde, sowie einen
universalistischen Anspruch, der es als Ziel sieht, allgemeing"ultige Aussagen
"uber Sprache insgesamt, nicht "uber ein bestimmte Sprache oder Sprachfamilie
zu machen.

Der Anspruch der psychologischen Ad"aquatheit kennzeichnet FG,
wie schon im Zusammenhang mit der zentralen Fragestellung erw"ahnt, als ein
mentalistisches Modell, das wie etwa die generative Syntaxtheorie ein Modell
f"ur die menschliche Sprachf"ahigkeit sein will, im Gegensatz zu rein
anwendungs- bzw. beschreibungsorientierten Ans"atzen wie HPSG.

FG geht im Gegensatz zur nativistischen Hypothese Chomskys
davon aus, dass sprachliche Universale nicht angeborenen Eigenschaften
entspringen sondern den Notwendigkeiten der menschlichen
Kommunikation\footnote{In diesem Sinne w"are etwa die Tatsache, dass alle
Sprachen eine Unterscheidung zwischen Funktions- und Inhaltsw"ortern haben, in
der Notwendigkeit begr"undet, die Inhalte einer sprachlichen "Ausserung
zueinander in Bezug zu setzen.} sowie der physischen und psychologischen
Konstitution des Menschen\footnote{Etwa eine Einschr"ankung der
Schachtelungstiefe von Nebens"atzen durch die begrenzten M"oglichkeiten des
menschlichen Kurzzeitged"achnisses.}, und kann damit als nicht-nativistisches
Modell charakterisiert werden.

\section{Hauptaufgabe linguistischer Forschung}

Ziel der Forschung im Rahmen der FG ist die Entwicklung eines
sprachunabh"angigen Formalismus zur Sprachbeschreibung. Dazu ist eine
ausgiebige Anpassung der bestehenden Formalismen an viele verschiedene Sprachen
n"otig (Dik 1991:248). In diesem Sinne ist die Sprachbeschreibung ein zentraler
Gegenstand der FG-Forschung.

Aus dem Anspruch der Formalisierbarkeit ergibt sich ein weiteres
Forschungsgebiet: Die Implementierung von FG auf einem Computer. Dik selbst hat
seit den 1980er Jahren vor allem auf diesem Gebiet gearbeitet. Diks eigene und
auch andere Implementierungen (etwa Samuelsdorff 1989) verwenden die
logikorientierte Programmiersprache Prolog (\emph{Programming in Logic}), die
sich aufgrund ihrer starken Orientierung an der Pr"adikatenlogik besonders zu
eignen schien, eine Umsetzung ist jedoch auch in jeder anderen
Programmiersprache m"oglich. Die Arbeiten in diesem Bereich konzentrieren sich
stark auf das Gebiet der Generierung und abstrakten Darstellung von
sprachlichen Ausdr"ucken, nicht auf die Verarbeitung (Parsing), die ebenso wie
die Generierung Teil der \emph{linguistic capacity} ist (siehe Abschnitt
\ref{kriterien} auf Seite \pageref{kriterien}).

Dar"uberhinaus legen die Forderungen nach pragmatischer und psychologischer
Ad"aquatheit (siehe Abschnitt \ref{kriterien} auf Seite \pageref{kriterien})
eine gewisse Offenheit und interdisziplin"are Zusammenarbeit nahe, wenn
Erkenntnisse relevanter F"acher wie Psychologie und Soziologie ber"ucksichtigt
werden sollen.

\section{Anwendungsorientiertheit und Anwendbarkeit}

\subsection{Vorteile}

Diks ``Functional Grammar'' scheint viele in anderen Modellen vernachl"assigte,
jedoch zur vollst"andigen Sprachbeschreibung wichtige Aspekte von Sprache zu
ber"ucksichtigen:

\begin{itemize}
  \item Eine Ber"ucksichtgung der Pragmatik -- etwa wie oben beschrieben bei
  einer Aufz"ahlung oder zur Beschreibung des Unterschiedes zwischen den
  Ausdr"ucken \emph{Buddhist Japanese} und \emph{Japanese Buddhist}.
  \item  Die zentrale Rolle der Semantik -- etwa bei der Zuordnung von
  \emph{who} an belebte und \emph{which} an unbelebte Mitspieler in einem
  Relativsatz oder bei den Selektionsbeschr"ankungen der Argumente in
  Pr"adikatrahmen.
  \item Die Unterscheidung von semantischen Rollen und grammatischen Relationen
  -- etwa zur Beschreibung von Aktiv-Passiv-Alternation ohne diese gegenseitig
  auseinander abzuleiten.
  \item Die Ber"ucksichtigung typologischer Eigenheiten vieler Sprachen -- etwa
  in Form der \emph{Semantic Function Hierachy} (SFH) zur Subjektivierbarkeit
  von Mitspielern mit bestimmten semantischen Rollen.
\end{itemize}

\subsection{M"ogliche Schw"achen und Probleme bei der Anwendung}

Die Verwendung merkmalsemantischer Primitive zur Selektionsbeschr"ankung
bestimmter Argumentpositionen in den UCS k"onnte in der Praxis zu den mit
diesem sematischen Modell bekannten Problemen f"uhren, etwa bei relationalen
Eigenschaften wie Verwandschaftsverh"altnissen, sowie bei Verben, graduellen
Unterschieden und Farben. Die Kodierung der Feinsemantik im Lexikon stellt
jedoch allgemein ein ungel"ostes Problem dar.

Das Vorgehen, Verletzungen der Selektionsbeschr"ankungen nicht als
ungrammatisch zu bezeichenen, sondern z.B. als Metapher zu behandeln stellt
eventuell eine Immunisierungstrategie dar (etwa wenn keine spezielle
Interpretationsstrategie ausgearbeitet wurde), die in diesem Fall Testbarkeit,
Anwendbarkeit und den wissenschaftlichen Wert des Modells verringern k"onnte.

Auch die aus der semantische Ausrichtung ergebende Konzentration auf
Introspektion und Elizitierung zur Datengewinnung f"ur die Bestimmung von
Selektionsbeschr"ankungen von Argumentpositionen in Pr"adikatrahmen k"onnte
Probleme verursachen und den wissenschaftstheoretischen Wert der gewonnenen
Daten schm"alern, da elizitierte und aus Introspektion gewonnene Daten durch
die vorgegebene Fragestellung leicht missinterpretiert werden k"onnen, etwa
wenn Einflussfaktoren, die "uber die Fragestellung hinaus gehen, nicht
ber"ucksichtigt werden.

Auch im Bereich der Operatoren zur zeitlichen Spezifizierung der Pr"adikation
gehen der universelle Anspruch und die praktischen Erfordernisse auseinander,
denn die auf dieser Ebene von Dik genannten Operatoren wie ``present'' und
``progressive'' sind keine universellen Kategorien, doch die UCS hat den
Anspruch vor Anwendung der Ausdrucksregeln sprachunabh"angig kodiert zu sein.

\section{Zusammenfassende Charakterisierung des Modells}

Zusammenfassend l"asst sich Simon C. Diks ``Functional Grammar'' als ein

\begin{itemize}
  \item{monostratales, nicht-derivationelles (s. Abschnitt \ref{monostratal} auf S. \pageref{monostratal})},
  \item{deszendentes (s. Abschnitt \ref{deszendent} auf S. \pageref{deszendent})},
  \item {induktives (s. Abschnitt \ref{induktiv} auf S. \pageref{induktiv})},
  \item {mentalistisches, nicht-nativistisches (s. Abschnitt \ref{mentalistisch} auf S. \pageref{mentalistisch})},
  \item {universalistisches (s. Abschnitt \ref{universalistisch} auf S. \pageref{universalistisch})},
  \item {pragmatikbasiertes, und in diesem Sinne funktionales (s. Abschnitt
  \ref{functional} auf S. \pageref{functional} sowie Abschnitt \ref{kriterien} auf S. \pageref{kriterien})} sowie
  \item {beschreibungsorientiertes und testbares, und in diesem Sinne formales (s. Abschnitt \ref{formal} auf S. \pageref{formal} sowie Abschnitt \ref{kriterien} auf S. \pageref{kriterien})}
\end{itemize}

linguistisches Modell charakterisieren.

% \newpage
% \thispagestyle{empty}
\section*{Bibliographie}

\begin{description}
\item [\textmd{\noun{Dik}}\textmd{,}]Simon C. 1991 {}``Functional Grammar'' In:
\noun{Droste}, F. \& John E. \noun{Joseph} (eds.) \emph{Theory and Grammatical
Description.} 247-274\emph{.}
\item [\textmd{\noun{Dik}}\textmd{,}]Simon C. 1997 {}``The Theory of Functional
Grammar''. Berlin, New York: Mouton de Gruyter.
\item [\textmd{\noun{Samuelsdorff}}\textmd{,}]Paul-O. 1989 {}``Simulation of a
Functional Grammar in Prolog'' In: \noun{Connolly}, John H. \& Simon C.
\noun{Dik} (eds.) \emph{Functional Grammar and the Computer.} 29-44\emph{.}\item
\end{description}

\end{document}
