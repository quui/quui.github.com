\documentclass[titlepage,a4paper]{article}

\usepackage[applemac]{inputenc}
\usepackage{ngerman}
\usepackage{tipa}
\usepackage{graphics,latexsym}
\usepackage{longtable}
\usepackage{rotating}

\author{Fabian Steeg}
\date{\today}
\title{\Large{Universit"at zu K"oln \\ Geographisches Institut }\\[5ex] 
\large{Proseminar Physische Geographie I\\ bei Alexandra Hilgers}\\[10ex] 
\LARGE{Protokoll zur Exkursion: Die Entwicklung der Terrassenlandschaft des 
Rheins in der n"ordlichen K"olner Bucht}\\[5ex]}

%fuer 1 1/2 den Wert 1.3 nehmen, doppelt 1.6!
\linespread{1.3}

\begin{document}
\begin{titlepage}
\maketitle
\tableofcontents
\listoffigures
%\listoftables
\thispagestyle{empty}
\end{titlepage}
\newpage

% ---------------------------------------

\section{Die n"ordliche K"olner Bucht}
    
    Die n"ordliche K"olner Bucht ist Teil der niederrheinischen Bucht und erstreckt 
    sich "uber die Fl"ache von Bonn bis zu einer gedachten Linie zwischen Grevenbroich 
    und Neuss. Im Exkursionsgebiet herrscht mild-gem"a"sigtes Klima mit einem "uber 
    das Jahr verteilten Niederschlag von 700 mm und einer Durchschnittstemperatur von 
    9,6\textdegree C (gemessen am K"oln-Bonner Flughafen).

\section{Standort 1: Worringer Bruch}

\subsection{Flussterrassen}

Flussterrasse sind ehemalige "Uberflutungsebenen, die nach einer Einschneidung des Flusses nicht mehr "uberflutet werden. Es lassen sich zwei Arten von Flussterrassen unterscheiden:

\begin{itemize}
\item{Akkumulationsterrassen (Schotterterrassen), am Rhein im Bereich des Niederrheins und in der K"olner Bucht}
\item{Erosionsterrassen (Felsterassen), am Rhein s"udlich von Bonn}
\end{itemize}

F"ur die Entstehung von Flussterrassen gibt es im Wesentlichen drei Gr"unde:

\begin{enumerate}
\item{Tektonische Aktivit"at (Anhebung eines Gebietes mit anschlie"sendem Einschneiden des Flusses)}
\item{Meerespiegelschwankungen (das Absinken des Meeresspiegels f"uhrt zu einer Absenkung der Erosionsbasis und damit zum Einschneiden des Flusses)}
\item{Wechsel zwischen Kalt- und Warmzeiten im Quart"ar (siehe folgender Abschnitt)}
\end{enumerate}
    
    \subsection{Terrassenbildung durch Wechsel von Kalt- und Warmzeiten im Quart"ar} \label{wild-fluss}
        
        Eine Ursache von Flussterrassenbildung liegt in den Wechseln zwischen Kalt- und Warmzeiten im Quart"ar (in den letzten 2,4 Mio. Jahren). Dabei wird in den Kaltzeiten im Gebirge durch Frostsprengung Frostschutt produziert, der im Fr"uhjahr von der Schneeschmelze die H"ange hinunter gesp"ult wird und vom Rhein flussabw"arts getrieben wird. Bei Bonn tritt nun der Rhein aus dem Gebirge in die Bucht und verliert so die notwendige Transportkraft und der Schutt wird abgelagert. Auf diese Weise wird eine Flussterrasse aufgeschottert.
        
        Wenn es nun zu einer Warmzeit kommt, nimmt die Frostsprengung ab und Vegetation stabilisiert die Gebirgsh"ange. Dies hat zur Folge, dass kein Schutt mehr abgesp"ult wird und sich der Fluss in den zuvor aufgeschotterten Boden eintiefen kann.
        
        In den Kaltzeiten war das Erscheinungsbild des Rheins durch verwilderte Flussl"aufe (engl. 'braided rivers') gekennzeichnet, die sich "uber das gesamte Flussbett in bis zu 16 km Breite erstreckten und ein Gewirr aus kleinen Flussl"aufen und Schotterinseln bildeten.
        
    \subsection{M"aanderbildung und -abschn"urung}
    
        In Warmzeiten - so auch jetzt im Holoz"an - ist das Erscheinungsbild der Rheins durch M"aanderbildung gekennzeichnet. Das Wort M"aander stammt  vom griechischen Namen 'Maiandros' f"ur den stark gewundenen t"urkischen Fluss 'B"uy"uk Menderes' und bezeichnet Flussschlingen, die im Mittel- und Unterlauf von Fl"ussen vorkommen. Beim Worringer Bruch (Rechtswert 25\-60\-200, Hochwert 56\-57\-500) handelt es sich um einen vor einigen 1000 Jahren abgeschn"urten M"aanderbogen des Rheins. Die Verlandung ist noch nicht weit fortgeschritten und es ist noch reichlich Altwasser vorhanden (siehe Abbildung \ref{bruch} auf Seite \pageref{bruch}). 

\begin{figure}
\begin{center}
\includegraphics{bruch.epsf}
\end{center}
\caption{Der Worringer Bruch} \label{bruch}
\end{figure}

        Zur Bildung der M"aander kommt es, wenn der Stromstrich\footnote{Die Stelle im Fluss mit der h"ochsten Flie"sgeschwindigkeit: An der Oberfl"ache, in der Mitte (da hier die Reibung am geringsten ist).} auf ein Hindernis trifft und das Ufer erodiert, auf das er gelenkt wird. Von dort tr"agt der Fluss Material zum gegen"uberliegenden Ufer ab. Auf der abgetragenen Seite entsteht so ein steiler Prallhang, gegen"uber ein flacher Gleithang, wodurch sich ein asymmetrisches Querprofil bildet.
        
      Zu einer M"aanderabschn"urung wie im Fall des Worringer Bruchs kommt es bei einer immer weiter fortschreitenden Verengung des M"aanderhalses durch die Seitenerosion des Flusses. Kommt es nun zu Hochwasserereignissen, sucht sich das Wasser den k"urzesten Weg und schn"urt den M"aandersporn ab (siehe Abbildung \ref{schnuerung} auf Seite \pageref{schnuerung}).  

\begin{figure}
\begin{center}
\includegraphics{schnuerung.epsf}
\end{center}
\caption{M"aanderhals und -sporn} \label{schnuerung}
\end{figure}

      
    \subsection{Bodentyp}
        
        Bei dem Bodentyp im Altarm handlt es sich um Auengley aus Auensand "uber Auenlehm mit einem 42 cm m"achtigen dunkelbraunen, feinsandigen Oberbodenhorizont (Ah), einem bis zu einer Tiefe von 85 cm reichenden rotbraunen, sandigen Unterbodenhorizont (Go) und einem grauen, lehmigen Untergrundhorizont (Gro). Das Profil ist komplett entkalkt und hat einen ph-Wert\footnote{Der ph-Wert ist der negative Logarithmus der Konzentration der positiven Wasserstoff-Ionen in der Bodenl"osung; je geringer der ph-Wert, desto saurer ist die L"osung.} von 4-5. Der wichtigste bodenbildende Prozess ist hier die Vergleyung. Dabei werden unter Sauerstoffmagel und andauernder Vern"assung die rostbraunen Eisen- und Manganhydroxide zu Manganoxiden reduziert (vgl. Bauer 2002).
                
\section{Standort 2: Kiesgrube in der Niederterrasse}

    Das durchschnittliche Niveau der in der letzten Kaltzeit (Weichsel-Kaltzeit) gebildeten Niederterrasse (Rechtswert 25\-56\-175, Hochwert 56\-63\-050) liegt bei ca. 40 m "uber Normalnull. Die Basis der Niederterrasse besteht vor allem aus Schotter und Kies, nach oben abgeschlossen wird sie von Sand und dar"uber von fluvialen Ablagerungen aus Hochflutlehm (siehe Abbildung \ref{nt1} auf Seite \pageref{nt1}). Bei der Aufschotterung der Terrasse wurde grunds"atzlich zuerst grobes Material abgelagert, feines zuletzt. Die Schr"agschichtung der Ablagerungen kann durch Str"omungsbewegungen erkl"art werden, die Wechsellagerung von Sand und Kies durch Aktivit"at und Inaktivit"at einzelner Flussl"aufe des verwilderten Flusssystems, das die Niederterrasse aufgeschottert hat.
    
\begin{figure}
\begin{center}
\includegraphics{nt.epsf}
\end{center}
\caption{Kiesgrube in der Niederterrasse} \label{nt1}
\end{figure}


\section{Standort 3: Rheinaue bei Zons}

    Das Niederterrassental hat zwischen K"oln und D"usseldorf eine Breite von 12-16 km. Die im Holoz"an gebildete Flussaue hat bei Zons (Rechtswert 25\-60\-050, Hochwert 56\-66\-000) eine Breite von ca. 500 Metern und liegt auf einer H"ohe von 35 m "uber Normalnull. 
            
    \subsection{Die Flussaue}
        Die Flussaue ist der tiefste Teil des Talbodens, sie schlie"st direkt an das Gerinne an und ist vom Grundwasser sowie von "Uberschwemmungen beeinflusst. Da die Flussstr"omung am Rand des Gerinnes am langsamsten ist, lagert der Fluss hier grobk"ornige Sedimente ab und bildet so einen Uferwall, an den sich eine Ebene anschlie"st, auf der feink"ornige Sedimente abgelagert werden.
                 
    \subsection{Bodentyp}
    
    Die "Uberschwemmungen der Aue lagern regelm"a"sig Hochflutlehm ab und sorgen so f"ur eine konstante Versorgung der Auenb"oden mit N"ahrstoffen, was diese zu sehr fruchtbaren B"oden macht. Diese konstante Versorgung mit neuem Material ist hier der wichtigste Prozess f"ur die Bodenbildung.
        
        Das Profil ist komplett kalkhaltig, der pH-Wert betr"agt 8. Die Profilabfolge lautet
         \begin{displaymath} A_{ap} - M_{a} - G\footnote{Zur Bedeutung der Abk"urzungen siehe Tabelle \ref{kuerzel} auf Seite \pageref{kuerzel}}\end{displaymath}
         Es handelt sich hier um braunen Auenboden aus sandigem Lehm. Als solcher verf"ugt er "uber viele Mittelporen und damit "uber eine hohe nutzbare Feldkapazit"at (nFK) d.h. er enth"alt viel pflanzenverf"ugbares Wasser. Da Lehm zudem viel Ton enth"alt zeichnet sich dieser Boden ausserdem durch eine hohe Sorptionsf"ahigkeit aus. Verbunden mit einer hohen biologischen Aktivit"at f"uhrt dies zu sehr fruchtbaren B"oden. Die Vegetation auf der Aue muss jedoch an die "Uberflutungen angepasst sein, so kann etwa die Silberweide bis zu 300 "uberflutete Tage im Jahr ertragen. Heute gibt es aufgrund der Nutzung der Auengebiete durch den Menschen in Mitteleuropa keine nat"urliche Auenvegetation mehr.
         
    \subsection{Hochwasserereignisse}
    
        Da die Auen unter anderem durch die regelm"a"sigen "Uberflutungen sehr fruchtbar sind werden sie vom Menschen besiedelt. Damit siedelt der Mensch aber im Retentionsraum (R"uckzugsgebiet f"ur Hochwasser) des Flusses und ist damit zwangsl"aufig mit Hochwasserereignissen in seinem Siedlungsraum konfrontiert. Zu Hochwasserereignissen kommt es vor allem im Winter. Die wesentlichen Gr"unde daf"ur sind:
        
        \begin{itemize} 
        \item{In den Mittelgebirgen kommt es zu heftigen Niederschl"agen}
        \item {Das Wasser kann durch die gefrorenen B"oden nur schlecht einsickern}         
        \item {Der geringe Wasserverbrauch durch die Vegetation (bedingt durch verminderte Verdunstung) f"uhrt zus"atzlich zu verst"arktem Wasserabfluss}
        \item {Bei Hochwasserereignissen akkumuliert sich schlie"slich das so abflie"sende Wasser und das Wasser der Schneeschmelz-Flutwelle}
        \end{itemize}
        
        Das Ausma"s der auf diese Weise entstehenden Hochwasserereignisse wird durch anthropogene Faktoren wie die Versiegelung von Bodenfl"achen (verhindert Einsickern des Niederschlags) und Flussbegradigungsma"snahmen (erh"ohen die Flie"sgeschwindigkeit des Flusses) verst"arkt. Inzwischen gibt es landesweit Programme zur Verminderung des Ausma"ses von Hochwasserereignissen:
        \begin{itemize} 
        \item Bachl"aufe werden renaturiert, um die Abflussgeschwindigkeit zu verringern
        \item Es werden neue "Uberflutungsgebiete geschaffen, etwa indem Deiche (die zuvor die Problematik lediglich verlagert haben) r"uckverlegt werden
        \item Regenwassersickergruben werden eingerichtet, die das Wasser dezentral zur"uckhalten, statt es gleich in den Rhein zu lassen
        \end{itemize}
\section{Standort 4: Binnend"une Wahler Berg}
    \subsection{Entstehung der Binnend"une Wahler Berg}
        Der Wahler Berg ist eine Binnend"une auf der Niederterrasse (Rechtswert 25\-57\-300, Hochwert 56\-65\-700) und entstand kurz nach Ausbruch des Laacher Sees vor 12880 Jahren\footnote{Diese pr"azise Datierung ist m"oglich durch den Fund von Laacher-See-Bims in der Niederterrasse} in der j"ungeren Dryas (Tundrenzeit). 
        \begin{figure}
\begin{center}
\includegraphics{duene.epsf}
\end{center}
\caption{Im Luv des Wahler Bergs} \label{duene}
\end{figure}
        Zu dieser Zeit war der Rhein ein verwildertes Flusssystem (siehe Abschnitt \ref{wild-fluss} auf Seite \pageref{wild-fluss}), am Standort befand sich damals eine Schotterinsel mit Tundrenvegetation. 
        
        Auf der Insel wurde immer im Winter Flugsand aus dem nun inaktiven westlichen Gerinne abgelagert (siehe Abbildung \ref{2gerinne} auf Seite \pageref{2gerinne}) 
        \begin{figure}
\begin{center}
\includegraphics{2gerinne.epsf}
\end{center}
\caption{Querprofil durch das verwilderte Flusssystem} \label{2gerinne}
\end{figure}
und bildete eine durch Vegetation gebundene Parabel- oder Bogend"une mit einer H"ohe von heute 5,3 m und einem Gef"alle von 14\textdegree{} im konkaven Luv und 6\textdegree{} im konvexen Lee (siehe Abbildung \ref{luvlee} auf Seite \pageref{luvlee}).
\begin{figure}
\begin{center}
\includegraphics{luvlee.epsf}
\end{center}
\caption{Luv und Lee einer Parabel- oder Bogend"une} \label{luvlee}
\end{figure}
 Die Parabeld"une ist die typische Form f"ur Binnend"unen in Mitteleuropa (im Gegensatz zu Sicheld"unen in W"usten). Inzwischen ist das Gebiet um die D"une als Natur- und Vogelschutzgebiet eingez"aunt, fr"uher wurde das Gebiet als Motocross-Strecke genutzt, was zu einer Ver"anderung der D"unenform f"uhrte (siehe linker Bereich von Abbildung \ref{duene} auf Seite \pageref{duene}).
    \subsection{Bodentyp}
        Der Bodentyp ist hier eine saure Braunerde aus Flugsand. Sie ist komplett entkalt und hat einen pH-Wert von 4. Die wichtigsten bodenbildenden Prozesse sind hier Verbraunung (Oxidation zu Goethit) und Verlehmung (Tonmineralneubildung durch Silikatverwitterung). Der Boden hat die Profilfolge
         \begin{displaymath} A_{h} - B_{v} - B_{v}C - C\footnote{Zur Bedeutung der Abk"urzungen siehe Tabelle \ref{kuerzel} auf Seite \pageref{kuerzel}}\end{displaymath}
        Der braune Ah-horizont hat eine M"achtigkeit von ca. 10 cm, der braun-rote Bv-Horizont von ca. 30 cm und der der etwas hellere rot-braune BvC-Horizont von ca. 25 cm. Der C-Horizont ist braun-gelb.  Da dieser Boden wenig Tonminerale enth"alt sickert Wasser schnell durch, was zu einem geringen Wassergehalt des Bodens f"uhrt.
        
\section{Standort 5: Rommerskirchener L"ossplatte}
    \subsection{Enstehung des L"oss auf der Mittelterrasse}
    Die aus Flussschotter aufgebaute Mittelterrasse (Rechtswert 25\-51\-450, Hochwert 56\-58\-800) entstand in der vorletzten Eiszeit (Saale-Eiszeit). Sie ist bis zu 20 m hoch und weist ein flachwelliges Relief auf. Auf der Mittelterrasse konnten sich in der letzten Eiszeit (Weichsel-Eiszeit) ausgewehte "aolische Sedimente von der Niederterrasse ablagern, da es hier nicht zu "Uberschwemmungen kam. Zudem wurde das Material von der K"altesteppenvegetation der Weichsel-Eiszeit stabilisiert. 

\begin{figure}
\begin{center}
\includegraphics{mt.epsf}
\end{center}
\caption{Die landwirtschaftlich genutzte Rommerskirchener L"ossplatte auf der Mittelterrasse} \label{mt}
\end{figure}

    
    Der abgelagerte L"oss hat sein Korngr"o"sen-Maximum im Grobschluff-Bereich und somit viele Mittelporen. Dadurch hat ein l"osshaltiger Boden eine hohe nutzbare Feldkapazit"at (nFK), d.h. er enth"alt viel pflanzenverf"ugbares Wasser. L"ossregionen werden deshalb vom Menschen landwirtschaftlich genutzt (siehe Abbildung \ref{mt} auf Seite \pageref{mt}), sie bilden die B"orde- und Gaulandschaften wie z.B. die Magdeburger B"orde und den Breisgau. Der L"oss auf der Mittelterrasse kann bei Niederschl"agen leicht verschl"ammen und so am Abgang zur Niederterrasse ein Kolluvium bilden.
    
    \subsection{Bodentyp}
    L"oss ist zun"achst kalkhaltig, wenn es aber zur Bodenbildung kommt ist der erste Prozess der der Entkalkung. Der Bodentyp ist hier eine Parabraunerde auf L"oss mit der Profilabfolge
    
    \begin{displaymath} A_{p} - A_{l}B_{t} - B_{t} - B_{v}\footnote{Zur Bedeutung der Abk"urzungen siehe Tabelle \ref{kuerzel} auf Seite \pageref{kuerzel}}\end{displaymath}
    
    Das Profil ist im oberen Bereich durch menschlichen Einfluss kalkhaltig, der pH-Wert betr"agt 5-6. Der wichtigste bodenbildende Prozess ist hier die Tonverlagerung oder Lessivierung (von franz. lessiver 'waschen'). 
    
    \subsection{Pseudo-Vergleyung}
    Bei zunehmender Tonverlagerung kann es zu einer so starken Verdichtung der Poren mit Ton kommen, dass kein Wasser mehr durchsickern kann. Dadurch kommt es zu einer Wasserstauung in Ober- und Unterbodenhorizont. Dabei ver"andert sich das Profil:
        \begin{center}
        \begin{tabular}{|rl|}
        \hline
        Parabraunerde & Pseudogley \\
        \hline %\hline
        Ah & Ah \\
        Al & \emph{Al-Sw} \\
        Bt & \emph{Bt-Sd} \\
        Bv & Bv \\
        \hline        
        \end{tabular}
        \end{center}
    Durch menschlichen Einfluss, vor allem durch Befahrung mit landwirtschaftlichem Ger"at, wird der Boden verdichtet, was die Pseudovergleyung f"ordert. Aufgrund der Kalk- und Tonauswaschung aus dem Oberboden sind Pseudogleye sauer und n"ahrstoffarm und damit schlecht f"ur die landwirtschaftliche Nutzung geeignet. M"ogliche Ma"snahmen zur Verbesserung von Pseudogley-B"oden sind tiefes Pfl"ugen, Kalkung und Humuszufuhr.
    
    \begin{table}
        \begin{center}
        
        \begin{tabular}{|rl|}
        \hline
        Mineralische Horizonte & \\
        \hline %\hline
        A & Terrestrischer Oberbodenhorizont \\
        B & Terrestrischer Unterbodenhorizont \\
        C & Terrestrischer Untergrundhorizont \\
        G & Semiterr. Bodenhorizont mit Grundwassereinfluss \\
        M & Bodenhorizont aus sedimetiertem Solummaterial \\
        S & Terr. Unterbodenhorizont mit Stauwassereinfluss \\
        \hline \hline
        Zusatzsymbole & \\
        \hline %\hline
        a & Auendynamik \\
        d & dicht (wasserstauend) \\
        p & gepl"ugt \\
        h & humos \\
        l & lessiviert,tonverarmt \\
        o & oxidiert \\
        r & reduziert \\
        v & verwittert,verbraunt,verlehmt \\
        w & wasserleitend \\
        \hline
        \end{tabular}
        
        \end{center}
        \caption{Verwendete K"urzel f"ur Bodenhorizonte (nach Bodenkundlicher Kartieranleitung, zitiert aus Wittmann (2004))}
        \label{kuerzel}
        \end{table}


\begin{figure}
       
        \begin{center}
        
        \includegraphics{quer.epsf}
        \\[0.5cm]
        \begin{tabular}{|p{1.7cm}||p{2cm}|p{2cm}|p{1.8cm}|p{1.5cm}|p{1.5cm}|}
        \hline
        Standort & 5 & 2 & 4 & 1 & 3 \\
        \hline \hline
        Lage & Mittel\-terrasse & Nieder\-terrasse & Nieder\-terrasse & Aue & Aue\\\hline
        Alter (Entstehung) & Saale-Kaltzeit & Weichsel-Kaltzeit & Weichsel-Kaltzeit & Holoz"an & Holoz"an \\\hline
        Substrat der Bodenbildung & L"oss & --- & Flug\-sand & Auensand & Hochflut\-lehm \\\hline
        Bodentyp & Para\-braunerde & --- & saure Braunerde & Auengley &  brauner Auenboden \\
        
                & Ap-AlBt-Bt-Bv & --- & Ah-Bv-BvC-C & Ah-Go-Gro & Aap-Ma-G \\ \hline
        Wichtigster Prozess & Lessivierung & --- & Ver\-braunung, Ver\-lehmung & Ver\-gleyung & --- (neues Material) \\
                        \hline
        Landw. Nutzbarkeit & sehr gut & --- & nicht gut & nicht gut & gut \\ \hline
        \end{tabular}
        
       
        \caption{"Ubersicht "uber die Exkursionsstandorte}
        \label{quer}
         \end{center}
        \end{figure}


     
\section{Querprofil des Exkursionsgebietes}
In Abbildung \ref{quer} auf Seite \pageref{quer} findet sich eine "Ubersicht "uber die Exkusrionsstandorte in einem Querprofil des Exkursionsgebietes.

%\newpage
\begin{figure}
\begin{thebibliography}{9999}

\bibitem{kompakt}
Bauer, J"urgen u.a. (2002) \textit{Physische Geographie kompakt}, Spektrum Akademischer Verlag, Heidelberg u.a.

\bibitem{boeden}
Wittmann, Otto u.a. (2004) \textit{Systematik der B"oden}, \textless http://www.uni-frankfurt.de/~relief/agb1/dbg86/kap1-123.htm\textgreater, 20.6.2004.

\end{thebibliography}
\end{figure}

\end{document}